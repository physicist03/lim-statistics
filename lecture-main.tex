\documentclass{article}

\usepackage{kotex}
\usepackage{graphicx}
\usepackage[affil-it]{authblk}
\usepackage{mathtools}
\usepackage{amssymb}
\usepackage{amsthm}
\usepackage{geometry}
\usepackage{fancyhdr}
\usepackage{braket}
\usepackage{cite}
\usepackage{cancel}
\usepackage{subcaption}
\usepackage{enumitem}
\usepackage{color}
\usepackage{chemformula}
\usepackage{physics}
\usepackage{hyperref}

\newcommand{\vp}{\varphi}
\newcommand{\ve}{\varepsilon}

\theoremstyle{definition}
\newtheorem{theorem}{Theorem}
\newtheorem{definition}[theorem]{Definition}
\newtheorem{example}[theorem]{Example}
\newtheorem{lemma}[theorem]{Lemma}
\newtheorem{axiom}[theorem]{Axiom}
\newtheorem{remark}[theorem]{Remark}
\newtheorem{problem}[theorem]{Problem}
\newtheorem{exercise}[theorem]{Exercise}

\counterwithin{equation}{section}
\counterwithin{theorem}{section}


\geometry{a4paper,left=2cm,right=2cm,top=2.4cm,bottom=2.4cm}

\linespread{1.3}

\title{\textsf{Introduction to Statistic}}
\author[1]{Written by Eun Taek Kang\thanks{email: etkang03@gmail.com}}
\affil[1]{Department of Physics, Sogang University, Seoul 04107, Korea}

\date{Summer 2025, Sogang University}

\begin{document}

\pagestyle{fancy}
    %... then configure it.
    \fancyhf{}
    % Set the header and footer for Even
    % pages but omit the zone (L, C or R)
    \fancyhead[R]{\textsf{Prof.\ Kyungpil Lim}}
    \fancyhead[L]{\textsc{Introduction to Statistic}}
    \fancyfoot[C]{\thepage}
    \fancyfoot[L]{\textbf{Sogang University}}
    \fancyfoot[R]{\textit{Department of Mathematics}}

\maketitle

\begin{abstract}
    본 문서는 서강대학교의 2025년 하계학기 통계학입문 강의의 내용을 필기하고 정리한 노트입니다. 모든 내용은 임경필 교수님의 강의를 바탕으로 작성되었습니다.
\end{abstract}

\newpage

\section{확률}

\subsection{확률의 정의}

\textbf{확률현상, 확률실험}

\begin{itemize}
    \item 확률현상(random phenomenon) : 불확실성에 의해서 좌우되는 현상
    \item 확률실험(random experiment) : 결과가 불확실성에 의해 좌우되는 실험이나 조사
\end{itemize}

\begin{example}[확률현상의 예]
    확률 현상의 예시들은 다음과 같다.

    \begin{enumerate}
        \item 동전이나 주사위를 던질 때 결과가 나타나는 현상
        \item 각 지역에서의 연 강수량의 변동 현상
        \item 수시로 달라지는 주가의 변동 현상
        \item 월 생산품에 포함되는 불량품(수)의 출현 현상
    \end{enumerate}
\end{example}

\noindent
\textbf{표본공간, 표본점, 사건}

\begin{enumerate}
    \item \textcolor{red}{표본공간}(sample space; $\Omega$) : 한 확률실험에서 얻어질 수 있는 가능한 모든 결과의 집합. 표본공간 = 전공간
    \item \textcolor{red}{표본점}(sample point; $\omega_i$) : 표본공간의 원소 하나
    \item \textcolor{red}{사건}(event) = \textcolor{red}{사상} : 표본공간의 부분집합
    \item \textcolor{red}{근원사건}(elementary event)= \textcolor{red}{단순사건}(simple event) : 표본점 하나하나로 이루어지는 사건
\end{enumerate}

\begin{example}
    확률실험에 대한 표본공간의 예시는 다음과 같다.

    \begin{enumerate}
        \item 하나의 동전을 던지고 윗면에 나타나는 상태를 관찰하는 확률실험: (단, Heads(앞면-그림), Tails(뒷면-금액))
        
        표본공간 : $\Omega = \{ H, T \}$

        \item 하나의 주사위를 던지고 윗면에 나타나는 점의 개수를 관찰하는 확률실험:
        
        표본공간 : $\Omega = \{1,2,3,4,5,6 \}$
    \end{enumerate}
\end{example}

\noindent
우리는 셀 수 있는 공간(countable)뿐 아니라 셀 수 없는 공간(uncountable)에서의 확률도 고려해야 한다. 이때 등장하는 것이 \textbf{기하적 확률}(Ex. 길이, 넓이, 부피)이다.\footnote{이런 부분은 실해석학(Real Analysis)의 공부에 도움이 된다.}

\vspace{4mm}\noindent
주사위를 굴려서 짝수가 나온다는 확률을 잴 때, \{2\}, \{4\}, \{6\}으로 쪼개지 \{2, 4\}, \{4, 6\}으로 굳이 쪼개어 생각하지 않는다. 따라서 이후 나올 배반사건으로 쪼개서 생각하는 것이 상당히 중요한 사고로 작용한다.

\newpage

\noindent
\textbf{사건의 결합(combination of events)}

\begin{enumerate}
    \item 여사건: $E_1^c$
    \item 합사건: $E_1 \cup E_2$
    \item 교사건: $E_1 \cap E_2 = E_1 E_2$\footnote{중간의 교집합 기호를 생략할 수 있다. Cartesian product가 아님에 유의하자.}
    \item 차사건: $E_1 - E_2$
    \item 무한가산합사건: $\bigcup_{i=1}^\infty E_i = \{ \omega \in \Omega \ | \ \exists \ i \in \mathbb{N}, \ \omega \in E_i \}$
    \item 무한가산교사건: $\bigcap_{i=1}^{\infty} E_i = \left\{ \omega \in \Omega \;\middle|\; \forall \ i \in \mathbb{N},\; \omega \in E_i \right\} $
\end{enumerate}

\noindent
\textbf{사전확률, 사후확률}

\begin{itemize}
    \item \textcolor{red}{사전확률} : 확률실험에서 모든 결과가 같은 가능성을 갖고 일어난다고 하자. 이 실험에서 일어날 수 있는 가능한 모든 결과의 수가 $N$, 또 한 사건 $E$를 일어나게 하는 결과의 수가 $n$이면, 이때 사건 $E$가 일어날 확률은 $P(E) = n/N$이다. 
    \item \textcolor{red}{사후확률} : 한 확률실험을 반복한 총 횟수를 $N$ 이라 하고, 또 그 중 사건 $E$를 일어나게 한 실험의 횟수를 $n$이라 하면, 이때 사건 $E$가 일어날 확률은 $P(E) = \lim_{N \rightarrow \infty} n/N$이다.
\end{itemize}

\noindent
\textbf{배반사건(mutually exclusive events)}

\begin{definition}
    두 사건 $E_1$과 $E_2$가 $E_1 \cap E_2 = \emptyset$을 만족할 때, $E_1$과 $E_2$는 상호 \textcolor{red}{배반사건}이라 한다.
\end{definition}

\noindent
경우의 수와 확률을 분석할 때, 상호 배반사건 관계에 있는 것들로 쪼개주는 것이 중요하다. 윷놀이 예시가 있는데 한 번쯤 분석해보도록 합시다.

\vspace{5mm}\noindent
\textbf{공리론적 확률}

\noindent
정의역때문에 설명이 어려워서 그냥 대충 받아들이고 가도 무방하나, 실해석학할 때 도움된다. (실사몇하;;)

\begin{definition}
    $\Omega$를 확률실험의 표본공간이라 하고, $\mathcal{F}$를 $\Omega$ 위에서 정의된 사건들의 `$\sigma$-집합체'라고 할 때, 집합함수 $P : \mathcal{F} \rightarrow \mathbb{R}$ ($\mathcal{F} \subset P(\Omega)$)가 다음 세 가지 조건을 만족할 때 $P$를 \textcolor{red}{확률함수}(probability function)라 하고, ($\Omega, \mathcal{F}, P$)를 \textcolor{red}{확률공간} (probability space)이라 한다.

    \begin{enumerate}
        \item[(A1)] $\forall E \in \mathcal{F}$, $P(E) \geq 0$
        \item[(A2)] 임의의 배반사건인 $\{E_n\}_{n=1}^\infty \subset \mathcal{F}$에 대하여,
        \begin{equation*}
            P \left( \bigcup_{n=1}^\infty E_n \right) = \sum_{n=1}^\infty P(E_n)
        \end{equation*} 
        \item[(A3)] $P(\Omega)=1$
    \end{enumerate}
\end{definition}

\noindent
상호 배반 합집합의 경우, (A2)의 $\cup$의 중간에 점을 찍어 표현하기도 한다. 이러한 공리론적 확률덕분에 배반사건에 대해서 합을 구하는 것은 상당히 쉽게 된다.

\newpage

\noindent
\textbf{$\sigma$-집합체($\sigma$-field), $\sigma$-집합대수($\sigma$-algebra)}

\begin{definition}
    $\Omega$를 확률실험의 표본공간이라 하고, $\mathcal{F}$를 $\Omega$의 부분 집합족이라 할 때, 다음 세가지 조건을 만족할 때 $\mathcal{F}$를 $\Omega$ 위에서의 \textcolor{red}{$\sigma$-집합체} 또는 \textcolor{red}{$\sigma$-집합대수}라 한다.

    \begin{enumerate}
        \item[(i)] $E \in \mathcal{F} \Rightarrow E^c \in \mathcal{F}$
        \item[(ii)] 임의의 $\mathcal{F}$의 가산무한 사건열 $\{E_n\}_{n=1}^\infty$에 대하여
        \begin{equation*}
            \bigcup_{n=1}^\infty E_n \in \mathcal{F}
        \end{equation*}

        \item[(iii)] $\emptyset \in \mathcal{F}$
    \end{enumerate}
\end{definition}

\begin{theorem}[$\sigma$-집합체의 기본 성질]
    $\mathcal{F}$를 $\Omega$ 위에서의 $\sigma$-집합체라 하자.

    \begin{enumerate}
        \item $\Omega \in \mathcal{F}$
        \item 임의의 $\mathcal{F}$의 가산무한 사건열 $\{E_n\}_{n=1}^\infty$에 대하여
        \begin{equation*}
            \bigcap_{n=1}^\infty E_n \in \mathcal{F}
        \end{equation*}
    \end{enumerate}
\end{theorem}

\begin{definition}[Borel $\sigma$-집합체]
    \(\mathbb{R}\)의 부분집합들의 집합족 \(\{(-\infty, x] \mid x \in \mathbb{R} \}\)를 포함하는 \(\mathbb{R}\) 위에서의 가장 작은 \(\sigma\)-집합체 \(\mathcal{B}\)를 \textcolor{red}{Borel \(\sigma\)-집합체}라고 한다. 
    Borel \(\sigma\)-집합체 \(\mathcal{B}\)는 \(\mathbb{R}\)의 모든 형태의 구간, 즉, 다음과 같은 형태의 구간을 포함한다.
    \begin{equation*}
        (-\infty, a], \quad (-\infty, a), \quad [a, \infty), \quad (a, \infty), \quad (a, b), \quad  (a, b], \quad [a, b), \quad [a, b]
    \end{equation*}
    여기서,
    \begin{equation*}
        (-\infty, a) = \bigcup_{n \in \mathbb{N}} \left(-\infty, a - \dfrac{1}{n} \right]
    \end{equation*}
\end{definition}

\begin{theorem}
    $(\Omega, \mathcal{F}, P)$가 확률공간일 때,

\begin{enumerate}
  \item 임의의 상호 배반인 $\mathcal{F}$의 유한 사건열 $\{E_1, E_2, \dots, E_n\}$에 대하여
  \[
  P(E_1 \cup E_2 \cup \cdots \cup E_n) = P\left( \bigcup_{i=1}^n E_i \right) = \sum_{i=1}^n P(E_i)
  \]

  \item $\forall E \in \mathcal{F}, \quad P(E^c) = 1 - P(E)$

  \item $\forall E \in \mathcal{F}, \quad 0 \leq P(E) \leq 1$

  \item $P(\emptyset) = 0$
\end{enumerate}
\end{theorem}

\begin{proof}
    ($\Omega, \mathcal{P}, P$) : 확률 공간

    \noindent
    (2) : $E \cup E^c = \Omega$이고, $E \cap E^c = \emptyset$ (상호 배반)이므로, (A2)에 의해 $P(E\cup E^c) = P(E) + P(E^c)$이고, (A3)에 의해 $P(\Omega) = 1$이다. $P(E \cup E^c) = P(\Omega)$이므로, $(E^c) = 1-P(E)$이 증명된다.

    \noindent
    (3)
\end{proof}

\noindent
\textcolor{blue}{\textbf{[Lecture 1 to here]}}

\end{document}