\documentclass{article}

\usepackage{kotex}
\usepackage{graphicx}
\usepackage[affil-it]{authblk}
\usepackage{mathtools}
\usepackage{amssymb}
\usepackage{amsthm}
\usepackage{geometry}
\usepackage{fancyhdr}
\usepackage{braket}
\usepackage{cite}
\usepackage{cancel}
\usepackage{subcaption}
\usepackage{enumitem}
\usepackage{color}
\usepackage{chemformula}
\usepackage{physics}
\usepackage{hyperref}

\newcommand{\vp}{\varphi}
\newcommand{\ve}{\varepsilon}

\newtheorem{theorem}{Theorem}
\newtheorem{definition}[theorem]{Definition}
\newtheorem{example}[theorem]{Example}
\newtheorem{lemma}[theorem]{Lemma}
\newtheorem{axiom}[theorem]{Axiom}
\newtheorem{remark}[theorem]{Remark}
\newtheorem{problem}[theorem]{Problem}
\newtheorem{exercise}[theorem]{Exercise}

\counterwithin{equation}{section}
\counterwithin{theorem}{section}


\geometry{a4paper,left=2cm,right=2cm,top=2.4cm,bottom=2.4cm}

\linespread{1.3}

\title{\textsf{Introduction to Statistic}}
\author[1]{Written by Eun Taek Kang\thanks{email: etkang03@gmail.com}}
\affil[1]{Department of Physics, Sogang University, Seoul 04107, Korea}

\date{Summer 2025, Sogang University}

\begin{document}

\pagestyle{fancy}
    %... then configure it.
    \fancyhf{}
    % Set the header and footer for Even
    % pages but omit the zone (L, C or R)
    \fancyhead[R]{\textsf{Prof.\ Kyungpil Lim}}
    \fancyhead[L]{\textsc{Introduction to Statistic}}
    \fancyfoot[C]{\thepage}
    \fancyfoot[L]{\textbf{Sogang University}}
    \fancyfoot[R]{\textit{Department of Mathematics}}

\maketitle

\begin{abstract}
    본 문서는 서강대학교의 2025년 하계학기 통계학입문 강의의 내용을 필기하고 정리한 노트입니다. 모든 내용은 임경필 교수님의 강의를 바탕으로 작성되었습니다.
\end{abstract}

\newpage

\section{여기에 섹션 제목을 입력해 주세요.}



\end{document}